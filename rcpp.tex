\documentclass[10pt, compress]{beamer}
\usetheme{metropolis}
\setsansfont{Avenir Light}
\setmainfont{Avenir Light}

\usepackage{booktabs}
\usepackage[scale=2]{ccicons}
\usepackage{dsfont}
\usepackage{fancyvrb}
\usepackage{caption}
\usepackage{hyperref}
\captionsetup[figure]{labelformat=empty}
\captionsetup{skip=0pt, belowskip=0pt}



\title{Rcpp}
\date{\today}
\author{Jacob Mortensen}
\institute{Simon Fraser University}

\begin{document}
  \maketitle
  
  \section{Introduction}
  \begin{frame}
    \frametitle{Introduction}
    Computation is a frequent bottleneck in statistics research
    \begin{itemize}
      \item R is often too slow
      \item C can become very complex, and can be very intimidating if you haven't dealt with it before
    \end{itemize}
    Enter Rcpp!
  \end{frame}

  \begin{frame}
    \frametitle{c++ vs. R}
    \begin{minipage}{0.45\textwidth}
      \centering
      R
      \newline
      \hrule
      \begin{enumerate}
        \item Weakly typed
        \item Interpreted
        \item 1-indexed 
      \end{enumerate}
    \end{minipage}
    \hfill
    \vrule
    \hfill
    \begin{minipage}{0.45\textwidth}
      \centering
      c++
      \newline
      \hrule
      \begin{enumerate}
        \item Strongly typed
        \item Compiled
        \item 0-indexed
      \end{enumerate}
    \end{minipage}
  \end{frame}

  \section{Advantages of Rcpp}
  \begin{frame}
  \frametitle{Advantages of Rcpp}
  Advantages of Rcpp:
  \begin{itemize}
    \item Speed comparable to pure C
    \item Much easier to use
    \begin{itemize}
      \item Memory management and data structures
      \item Syntactic ``sugar''
      \item Can use inline code or an external file
    \end{itemize}
  \end{itemize}
  \end{frame}

\begin{frame}[fragile]
  \frametitle{Memory management and data structures}
    R:
    \begin{Verbatim}[fontsize=\footnotesize]
output <- matrix(0, nrow = sigma2length*mu2length, ncol = 5)
    \end{Verbatim}
    C:
    \begin{Verbatim}[fontsize=\footnotesize]
double* output = (double*) malloc(5*mu2len*sigma2len*sizeof(double));
    \end{Verbatim}
    Rcpp:
    \begin{Verbatim}[fontsize=\footnotesize]
NumericMatrix output(sigma2length*mu2length, 5);
    \end{Verbatim}

\end{frame}

\begin{frame}[fragile]
  \frametitle{Memory management and data structures}
  Data structures available through Rcpp:
  \begin{itemize}
    \item NumericVector
    \item IntegerVector
    \item CharacterVector
    \item LogicalVector
  \end{itemize}
  Equivalent data structures exist for R matrices.
  All of these data structures are pointers, and so nothing is copied unless needed.
  Memory is allocated and freed automatically.
\end{frame}

\begin{frame}[fragile]
\frametitle{Memory management and data structures}
Initialize vector:
\begin{verbatim}
NumericVector x(10);
NumericVector x = NumericVector::create(1,2,3,4,5);
\end{verbatim}
Initialize matrix:
\begin{verbatim}
NumericMatrix x(10, 10);
\end{verbatim}
\end{frame}

\begin{frame}[fragile]
\frametitle{Memory management and data structures}
Access and set vector elements:
\begin{Verbatim}[fontsize=\small]
x[0];
x[1] = 4.2;
\end{Verbatim}
Access and set matrix elements:
\begin{Verbatim}[fontsize=\small]
x(0,0); 
x(0,1) = 42;
x(0, _); // returns first row of matrix as a NumericVector
\end{Verbatim}
\textbf{Note that these data structures are all indexed by zero!}
\end{frame}

\begin{frame}[fragile]
\frametitle{Syntactic sugar}
Many functions from R have been vectorised and implemented in c++
\begin{Verbatim}[fontsize=\small]
NumericVector euclid_dist(double x, NumericVector ys) {
  return sqrt(pow((x - ys), 2));
}
\end{Verbatim}
\end{frame}
\begin{frame}[fragile]
\frametitle{Syntactic sugar}
R functions implemented in Rcpp:
\begin{itemize}
  \item Arithmetic and logical operators: *, +, -, /, pow, <, <=, >, >=, ==, !=, !. 
  \item Math functions: abs(), beta(), exp(), gamma(), \dots
  \item Summary functions: mean(), min(), max(), sum(), sd() and var()
  \item d/p/q/r for all standard distributions in R
\end{itemize}
\end{frame}
  
\plain{Code examples}

\section{Rcpp Armadillo}
\begin{frame}
  \frametitle{Rcpp Armadillo}
  RcppArmadillo is a linear algebra library, so anytime you need to 
  \begin{itemize}
    \item invert a matrix
    \item perform matrix algebra
    \item decompose a matrix
  \end{itemize}
  this is where you should look.
\end{frame}

\begin{frame}[fragile]
\frametitle{Rcpp Armadillo Data Structures}
Use \texttt{arma::colvec} and \texttt{arma::mat} in place of \texttt{NumericVector} and \texttt{NumericMatrix}. 

Can convert between the two by using:
\begin{Verbatim}[fontsize=\small]
NumericVector x = wrap(y);
arma::colvec y = as<arma::colvec>(x);
\end{Verbatim}
and
\begin{Verbatim}[fontsize=\small]
NumericMatrix x = wrap(y);
arma::mat y = as<arma::mat>(x);
\end{Verbatim}

\end{frame}

\begin{frame}
  \frametitle{Rcpp Armadillo Functions}
  If you need to do any kind of linear algebra operation, it is probably available, and the \textbf{\href{http://arma.sourceforge.net/docs.html}{documentation is really good}}
\end{frame}

\plain{Code examples}

\section{Misc}
\begin{frame}[fragile]
  \frametitle{Using Other C++ Libraries}
  \begin{Verbatim}
// [[Rcpp::depends(RcppGSL)]]

#include <RcppGSL.h>
#include <gsl/gsl_rng.h>
#include <gsl/gsl_randist.h>
  \end{Verbatim}
\end{frame}

\begin{frame}[fragile]
  \frametitle{Calling R Functions in Rcpp}
\begin{Verbatim}[fontsize=\small]
RObject callWithOne(Function f) {
  return f(1);
}
> callWithOne(function(x) x + 1)
# [1] 2
\end{Verbatim}

\end{frame}

\begin{frame}
  \frametitle{Resources}
  \begin{itemize}
    \item \href{http://adv-r.had.co.nz/Rcpp.html}{Hadley Wickham's Advanced R Site}
    \item \href{http://www.rcpp.org/}{Official Rcpp Website}
    \item \href{http://arma.sourceforge.net/docs.html}{Armadillo Documentation}
  \end{itemize}
\end{frame}
\end{document}
